% Created 2022-01-25 ter 15:28
% Intended LaTeX compiler: pdflatex
\documentclass[letterpaper, 11pt]{article}
                      \usepackage{lmodern} % Ensures we have the right font
\usepackage[T1]{fontenc}
\usepackage[utf8]{inputenc}
\usepackage{graphicx}
\usepackage{amsmath, amsthm, amssymb}
\usepackage[table, xcdraw]{xcolor}
\definecolor{bblue}{HTML}{0645AD}
\usepackage[colorlinks]{hyperref}
\hypersetup{colorlinks, linkcolor=blue, urlcolor=bblue}
\usepackage{titling}
\setlength{\droptitle}{-6em}
\setlength{\parindent}{0pt}
\setlength{\parskip}{1em}
\usepackage[stretch=10]{microtype}
\usepackage{hyphenat}
\usepackage{ragged2e}
\usepackage{subfig} % Subfigures (not needed in Org I think)
\usepackage{hyperref} % Links
\usepackage{listings} % Code highlighting
\usepackage[top=1in, bottom=1.25in, left=1.55in, right=1.55in]{geometry}
\renewcommand{\baselinestretch}{1.15}
\usepackage[explicit]{titlesec}
\pretitle{\begin{center}\fontsize{20pt}{20pt}\selectfont}
\posttitle{\par\end{center}}
\preauthor{\begin{center}\vspace{-6bp}\fontsize{14pt}{14pt}\selectfont}
\postauthor{\par\end{center}\vspace{-25bp}}
\predate{\begin{center}\fontsize{12pt}{12pt}\selectfont}
\postdate{\par\end{center}\vspace{0em}}
\titlespacing\section{0pt}{5pt}{5pt} % left margin, space before section header, space after section header
\titlespacing\subsection{0pt}{5pt}{-2pt} % left margin, space before subsection header, space after subsection header
\titlespacing\subsubsection{0pt}{5pt}{-2pt} % left margin, space before subsection header, space after subsection header
\usepackage{enumitem}
\setlist{itemsep=-2pt} % or \setlist{noitemsep} to leave space around whole list
\author{Vinicius Faria}
\date{\today}
\title{Geometria analítica\\\medskip
\large \emph{Anotações Práticas}}
\hypersetup{
 pdfauthor={Vinicius Faria},
 pdftitle={Geometria analítica},
 pdfkeywords={},
 pdfsubject={},
 pdfcreator={Emacs 27.2 (Org mode 9.5.2)}, 
 pdflang={English}}
\begin{document}

\maketitle
\tableofcontents


\section{Espaços vetoriais e subvetoriais}
\label{sec:org159721d}
\subsection{Espaços vetoriais}
\label{sec:org9b8007f}
Dado um conjunto V, V é um espaço vetorial real caso satisfazer as condições:
\textbf{OBS:} Nas equações abaixo, \(\forall x,y,z \in V\) e \(\forall \alpha , \beta \in \realbb{R}\)
\begin{enumerate}
\item \(x + y = y + x\) (Associatividade)
\item \((x+y)+z = x + (y+z)\) (Comutatividade)
\item \(\exists \theta \in V \ / \ x + \theta = \theta + x = x\) (Existencia do vetor nulo)
\item \(-x \in V \ / \ x + (-x) = \theta\) (Elemento simétrico da soma)
\item \(\alpha (x+y) = \alpha x + \alpha y}\) (Distribuitividade)
\item \((\alpha + \beta)x = \alpha x + \beta x\) (Distruibitividade)
\item \((\alpha \beta) x = \alpha (\beta x)\) (Associatividade)
\item \(1x = x\) (Elemento neutro).
\end{enumerate}

Um espaço vetorial que contém os numeros complexos é denominado \textbf{espaço vetorial complexo}

A partir das expressões acima, é possivel extrair as afirmações:
\begin{enumerate}
\item \(\alpha \theta = \theta\)
\item \(0 \dotc x = \theta\)
\item \(\alpha x = \theta , então \ \alpha = 0 \ ou \ x = \theta\)
\item \((-\alpha )x = \alpha(-x)\)
\end{enumerate}

Alguns exemplos de espaços vetoriais, considerando operações normais:
\begin{itemize}
\item \(V = K_n(x) = {P_r(x) / r \le n}\)
\item \(V = C[a,b]\) \textbf{OBS:} \((f+g)(x) = f(x) + g(x)\) e \((\alpha f)(x) = \aplha f(x)\)
\item \(V = \realbb{R}^n\)
\item \(V = M_{mxn}\)
\end{itemize}

Não são espaços vetoriais:
\begin{itemize}
\item \(V = \realbb{Z}\)
\item \(V =\) Conjunto de polinômios de grau 3
\item Alguns conjuntos com operações diferentes do normal. Ex: \(\alpha x = \alpha (x_1, x_2) = (\alpha ^2 x_1, \alpha x_2)\)
\end{itemize}

\subsection{Subespaços vetoriais}
\label{sec:org0432817}
\subsubsection{Definições}
\label{sec:org12d2fe6}
Dentro de um espaço vetorial V, há subconjuntos W que são espaços vetoriais menores, contidos em V. \textbf{Todo espaço vetorial possui 2 subsespaços triviais:} Sub. nulo e ele mesmo.
Critérios para subespaços vetoriais:
\begin{enumerate}
\item \(\theta \in W\)
\item \(\forall x, y \in W \rightarrow x + y \in W\)
\item \(\forall \alpha \in \realbb{R}, \forall x \in W \rightarrow \alpha x \in W\)
\end{enumerate}
São subespaços:
\begin{itemize}
\item Soluções lineares homogêneas
\item Qualquer sistema que adote multiplicações/adições usual com a presença do vetor nulo
\end{itemize}
Não são subespaços:
\begin{itemize}
\item Geralmente sistemas com alguma multiplicação/adição fora do comum ou muito específicas (ex. Conjunto de polinomios de terceiro grau)
\item \(u + v = (u_1 , u_1^2) + (v_1, v_1^2)\)
\end{itemize}
\subsubsection{Teoremas}
\label{sec:orgba17d65}
\textbf{Interseccão de subespaços}. Dado \(W_1 \  e \  W_2\) subespaços vetorias de V, \(W_1 \cap W_2\) também é subconjunto de V. Ex - Matriz triangulares inferiores e superiores.

\textbf{Soma de subespaços}. Dado \(W_1 \  e \  W_2\) subespaços vetoriais de V, \(W_1 + W_2\) também é subconjunto de V. Caso \(W_1 \cap W_2 = \theta\), a soma é chamada de \textbf{soma direta}, denotada por \(W_1 \oplus W_2\)


\section{Combinação linear}
\label{sec:org85c55fe}
Dado um espaço vetorial V, o vetor \(v \in V\) é considerado a combinação linear de \(v_1 + v_2 + ... v_n\) caso houver escalares \(\alpha_1 ... \alpha_n\) tais que:

\begin{center} $v = \alpha_1v_1 + ... + \alpha_nv_n$ \end{center}

Em exercícios envolvendo combinação linear, dado v e \(v_1 ... v_n\), colocar tudo em um sistema e resolver (escalonando ou manualmente). \textbf{Caso o sistema for possível e determinado, é uma combinação linear,
caso contrário (por desigualdades depois de substituir valores, por exemplo) não é uma combinação linear}.

Como exemplo, utilizando os vetores \(v_1 = (1,-1,3) \ e \ v_2 = (2,3,0)\) para a combinação linear, verificar se podem ser combinados para formar: \(u = (3,3,3) \ e \ v=(-2,-8,6)\).

\begin{itemize}
\item \(u\) não pode ser formado, pois, dependo da maneira que o sistema for resolvido \(\alpha_2 = 1\) ao mesmo tempo que \(\alpha_2 = \frac{4}{3}\), classificando o sistema como impossível.

\item Agora, \(v\) é totalmente possível, pois ao escalonar ou tentar resolver o sistema claramente chegamos em \(\alpha_1 = 2\) e \(\alpha_2 = -2\).
\end{itemize}
\end{document}