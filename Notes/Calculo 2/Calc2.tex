% Created 2022-01-23 dom 20:53
% Intended LaTeX compiler: pdflatex
\documentclass[letterpaper, 11pt]{article}
                      \usepackage{lmodern} % Ensures we have the right font
\usepackage[T1]{fontenc}
\usepackage[utf8]{inputenc}
\usepackage{graphicx}
\usepackage{amsmath, amsthm, amssymb}
\usepackage[table, xcdraw]{xcolor}
\definecolor{bblue}{HTML}{0645AD}
\usepackage[colorlinks]{hyperref}
\hypersetup{colorlinks, linkcolor=blue, urlcolor=bblue}
\usepackage{titling}
\setlength{\droptitle}{-6em}
\setlength{\parindent}{0pt}
\setlength{\parskip}{1em}
\usepackage[stretch=10]{microtype}
\usepackage{hyphenat}
\usepackage{ragged2e}
\usepackage{subfig} % Subfigures (not needed in Org I think)
\usepackage{hyperref} % Links
\usepackage{listings} % Code highlighting
\usepackage[top=1in, bottom=1.25in, left=1.55in, right=1.55in]{geometry}
\renewcommand{\baselinestretch}{1.15}
\usepackage[explicit]{titlesec}
\pretitle{\begin{center}\fontsize{20pt}{20pt}\selectfont}
\posttitle{\par\end{center}}
\preauthor{\begin{center}\vspace{-6bp}\fontsize{14pt}{14pt}\selectfont}
\postauthor{\par\end{center}\vspace{-25bp}}
\predate{\begin{center}\fontsize{12pt}{12pt}\selectfont}
\postdate{\par\end{center}\vspace{0em}}
\titlespacing\section{0pt}{5pt}{5pt} % left margin, space before section header, space after section header
\titlespacing\subsection{0pt}{5pt}{-2pt} % left margin, space before subsection header, space after subsection header
\titlespacing\subsubsection{0pt}{5pt}{-2pt} % left margin, space before subsection header, space after subsection header
\usepackage{enumitem}
\setlist{itemsep=-2pt} % or \setlist{noitemsep} to leave space around whole list
\author{Vinicius Faria}
\date{\today}
\title{Cálculo 2\\\medskip
\large \emph{Anotações Práticas}}
\hypersetup{
 pdfauthor={Vinicius Faria},
 pdftitle={Cálculo 2},
 pdfkeywords={},
 pdfsubject={},
 pdfcreator={Emacs 27.2 (Org mode 9.5.2)}, 
 pdflang={English}}
\begin{document}

\maketitle
\tableofcontents


\section{Funções Vetoriais}
\label{sec:org5b359d6}
Funções cuja imagem gerada é um vetor. \(\realbb{R}^2 \ e \ \realbb{R}^3\) são chamadas \textbf{funções coordenadas} e t é denominado \textbf{variável livre}

\begin{center}   $\alpha(t) = (2t+1, 1-t)$ \end{center}

\subsection{Esboço de curvas}
\label{sec:orga7ec006}
Para esboçar curvas, isolar o x e y (e potencialmente z) em função de t, formando a imagem da curva. A curva abaixo descreve um círculo.

\begin{center} $\begin{cases} x(t) = \sin (t) \\ y(t) = \cos(t) \end{cases}$ \end{center}

\subsection{Equações paramétricas de retas}
\label{sec:org372332e}
Dado uma curva que passa por A e B, é possível definir sua função utilizando:
\begin{center} $\alpha (t) = (1-t)A + tB = A + t(B-A)$ \end{center}

Interpretação geométrica: parametrização da reta que contém o ponto A e é parelela ao vetor não nulo (B-A)

\subsection{Parametrizações}
\label{sec:org9972f03}
Uma função vetorial é uma \uline{parametrização} da curva que é a imagem da função. Uma curva pode ser parametrizada de várias maneiras, como círculos.

\subsection{Limites e Continuidade}
\label{sec:orge684c6b}
Resumidamente, para calcular limites de funções vetoriais, basta calcular o limite de cada função de coordenada separadamente.

Da mesma forma, para verificar a continuidade da função vetorial, basta conferir que todas as funções de coordenadas sejam contínuas.

\textbf{OBS:} Uma função é contínua quando em seu domínio:
\begin{itemize}
\item Não possui assíntotas verticais
\item Não possui "furos"
\item Não possui "pulos"
\end{itemize}
\end{document}